\documentclass[a4paper, 11pt]{article}


%%%% Packages
%% Critical packages
\usepackage{graphicx}
\usepackage{amsmath}
\usepackage{amssymb}
\usepackage{amsthm}
\usepackage{fancyhdr}
\usepackage{enumerate}
\usepackage{float}
\usepackage{geometry}
\usepackage{newpxmath}
\usepackage{newpxtext}
\usepackage{lipsum}
\usepackage{enumitem}

%% Situational packages
\usepackage{mathtools}	% Formatting vertically centred colon (e.g. in \defeq).


%%%% Page formatting
\geometry{left = 66pt}
\geometry{right = 66pt}
\geometry{top = 54pt}
\geometry{bottom = 74pt}
\geometry{headsep = 14pt}
\setlength{\headheight}{14pt}

\setlength{\parindent}{0pt}
\setlength{\parskip}{10truept}


%%%% Custom commands
%% Formatting
\newcommand{\hln}{\vspace{-7mm}\begin{flushleft}\mbox{}\hrulefill\mbox{}\end{flushleft}\vspace{-7mm}}
% Place a horizontal line with less spacing below with less spacing.
% Not sure why this works.

\newcommand{\ds}{\displaystyle}
% Quick force displaystlye

\newcommand{\hsurround}[1]{\hln #1 \vspace{-6pt} \hln}
% Place a horizontal line above and below.

%% Common sets
\newcommand{\R}{\mathbb{R}}
\newcommand{\I}{\mathbb{I}}
\newcommand{\C}{\mathbb{C}}
\newcommand{\Z}{\mathbb{Z}}
\renewcommand{\P}{\mathcal{P}}
\newcommand{\Q}{\mathbb{Q}}
\newcommand{\N}{\mathbb{N}}
\newcommand{\F}{\mathbb{F}}

%% Boldface (e.g. for vectors)
\renewcommand{\bf}[1]{\textbf{#1}}
\renewcommand{\vec}[1]{\bf{#1}}

%% Common operators
\newcommand{\diag}[1]{\textrm{diag}\left(#1\right)}
\newcommand{\re}[1]{\textrm{Re}\left(#1\right)}
\newcommand{\imag}[1]{\textrm{Im}\left(#1\right)}
\newcommand{\cmod}[1]{\ (\mathrm{mod}\ #1)}
\DeclarePairedDelimiter\norm{\lVert}{\rVert}%

%% Common characters
\newcommand{\defeq}{\vcentcolon=}
\newcommand{\eqdef}{=\vcentcolon}
\newcommand{\st}{\ensuremath{\ s.t. \ }}

%% Theorems
\theoremstyle{plain}
\newtheorem*{theorem}{Theorem}

\theoremstyle{definition}
\newtheorem*{definition}{Definition}
\newtheorem*{example}{Example}

%% Fancy Headers and Footers

\pagestyle{fancy}
\fancyhead[L]{MATH2922}
\fancyfoot[R]{\tiny{\copyright \the\year{} The University of Sydney}}

\fancypagestyle{firstpage}{
	\pagestyle{fancy}
	\fancyhead{}
}

\begin{document}

	%%%%% Special header and title.

	%% First page has special settings.
	\thispagestyle{firstpage}
	
	%% Custom "Header"
	\begin{center}
	
		\vspace*{-34pt}
		\textsc{The University of Sydney}\\
		
		\textsc{School of Mathematics and Statistics}
		\vspace{-3mm}
		
		\hsurround{MATH2922: Linear and Abstract Algebra (Advanced) \hfill Semester 1, \the\year{}}
		
	\end{center}
	
	%% Title and explanation
	\begin{center}
	
	{ \Large \textbf{Introductory Module X -- Tutorial/Solutions} }
	
	\textit{Explanatory, introductory text goes here. }
	\end{center}
	
%% Questions

\begin{enumerate}[wide, label = \textbf{\arabic*.}]

	\item Let $\F$ be a field.
	
	\begin{enumerate}[leftmargin=1.2cm]
	\item Let $\lambda \in \F$. Show that $0\lambda = 0$.
	\end{enumerate}
	
\end{enumerate}

%% Solutions

\begin{enumerate}[wide, label = \textit{Solution to} \textbf{\arabic*.}]

	\item \phantom{.}
	
	\begin{enumerate}[leftmargin=1.2cm]
	\item As $0$ is an additive identity in $\F$, using the distributive law and part (a), $0 = \lambda (0 + 0) = \lambda0 + \lambda0$. Hence, $0\lambda = 0$ by the Cancellation law (that is, by adding $-\lambda0$ to both sides). Hence $0\lambda = 0$.
	\end{enumerate}
	
\end{enumerate}

\end{document}
